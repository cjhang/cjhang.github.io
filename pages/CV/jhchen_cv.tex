% Based on the template of https://github.com/fizixmastr 

\documentclass[A4,11pt]{article}
%\documentclass[letterpaper,11pt]{article} %For use in US
\usepackage{latexsym}
\usepackage[empty]{fullpage}
%\usepackage{titlesec}
\usepackage{marvosym}
\usepackage[usenames,dvipsnames]{color}
\usepackage{verbatim}
\usepackage{enumitem}
\usepackage[hidelinks]{hyperref}
\usepackage[english]{babel}
\usepackage{tabularx}
\usepackage{tikz}
\usepackage{titlesec}
%\input{glyphtounicode}
\hypersetup{
    colorlinks=true,
    urlcolor=RoyalPurple,
    }


\begin{comment}
  comments
\end{comment}


%-----FONT OPTIONS-------------------------------------------------------------
\begin{comment}
The font of the document will impact not just how readable it is, but how it is
perceived. In the "The Craft of Scientific Writing" by Michael Alley, shares a
common fonts for publication as well as their use. I have chosen to use
Palatino for its legibility, some others are given below. There is far too much
about typography to discus here. Note: serif fonts have short projecting
strokes, sans-serif fonts are sans (without) these strokes.
\end{comment}


% serif
 %\usepackage{palatino}
  %\usepackage{times} %This is the default as well
%\usepackage{charter}
\usepackage{newtxtext,newtxmath}

% sans-serif
% \usepackage{helvet}
% \usepackage[sfdefault]{noto-sans}
% \usepackage[default]{sourcesanspro}

%-----PAGE SETUP---------------------------------------------------------------

% Adjust margins
\addtolength{\oddsidemargin}{-1cm}
\addtolength{\evensidemargin}{-1cm}
\addtolength{\textwidth}{2cm}
\addtolength{\topmargin}{-1cm}
\addtolength{\textheight}{2cm}

% Margins for US Letter size
%\addtolength{\oddsidemargin}{-0.5in}
%\addtolength{\evensidemargin}{-0.5in}
%\addtolength{\textwidth}{1in}
%\addtolength{\topmargin}{-.5in}
%\addtolength{\textheight}{1.0in}

\urlstyle{same}

\raggedbottom
\raggedright
\setlength{\tabcolsep}{0cm}

% Sections formatting
\titleformat{\section}{
  \vspace{-4pt}\scshape\raggedright\large
}{}{0em}{}[\color{black}\titlerule \vspace{-5pt}]

% Ensure that .pdf is machine readable/ATS parsable
%\pdfgentounicode=1

%-----CUSTOM COMMANDS FOR FORMATTING SECTIONS----------------------------------
\newcommand{\CVItem}[1]{
  \item\small{
    {#1 \vspace{-2pt}}
  }
}

\newcommand{\CVSubheading}[4]{
  \vspace{-2pt}\item
    \begin{tabular*}{0.97\textwidth}[t]{l@{\extracolsep{\fill}}r}
      \textbf{#1} & #2 \\
      \small#3 & \small #4 \\
    \end{tabular*}\vspace{-7pt}
}

\newcommand{\CVSubSubheading}[2]{
    \item
    \begin{tabular*}{0.97\textwidth}{l@{\extracolsep{\fill}}r}
      \text{\small#1} & \text{\small #2} \\
    \end{tabular*}\vspace{-7pt}
}

\newcommand{\CVSubItem}[1]{\CVItem{#1}\vspace{-4pt}}

\renewcommand\labelitemii{$\vcenter{\hbox{\tiny$\bullet$}}$}

\newcommand{\CVSubHeadingListStart}{\begin{itemize}[leftmargin=0.5cm, label={}]}
% \newcommand{\resumeSubHeadingListStart}{\begin{itemize}[leftmargin=0.15in, label={}]} % Uncomment for US
\newcommand{\CVSubHeadingListEnd}{\end{itemize}}
\newcommand{\CVItemListStart}{\begin{itemize}}
\newcommand{\CVItemListEnd}{\end{itemize}\vspace{-5pt}}

\newenvironment{paperlist}
{ \begin{itemize}[leftmargin=0.8cm, label={$\bullet$}]
    \setlength{\itemsep}{1pt}
    \setlength{\parskip}{1pt}
    \setlength{\parsep}{1pt}     }
{ \end{itemize}                  } 


%------------------------------------------------------------------------------
% CV STARTS HERE  %
%------------------------------------------------------------------------------
\begin{document}

%-----HEADING------------------------------------------------------------------
\begin{comment}
In Europe it is common to include a picture of ones self in the CV. Select
which heading appropriate for the document you are creating.
\end{comment}

\begin{minipage}[c]{0.05\textwidth}
\-\
\end{minipage}
\begin{minipage}[c]{0.2\textwidth}
\begin{tikzpicture}
    \clip (0,0) circle (1.75cm);
    \node at (0.2,-.5) {\includegraphics[width = 5cm]{Jianhang.jpg}}; 
    % if necessary the picture may be moved by changing the at (coordinates)
    % width defines the 'zoom' of the picture
\end{tikzpicture}
\hfill\vline\hfill
\end{minipage}
\begin{minipage}[c]{0.6\textwidth}
    \textbf{\Huge Jianhang CHEN} \\
    % \scshape sets small capital letters, remove if desired
    %\small{+49 1627363810} \\
    \vspace{-0.5pt}
    %\small{Nationality: China} \\
    %{Citizenship: China} \\
    \href{mailto:cjhastro@gmail.com}{\underline{cjhastro@gmail.com}}\\
    % Be sure to use a professional *personal* email address
    %\href{https://www.linkedin.com/in/charles-rambo/}{\underline{linkedin.com/in/charles-rambo}} \\
    \href{https://cjhang.github.io}{\underline{https://cjhang.github.com}}\\
    % you should adjust you linked in profile name to be professional and recognizable
    \href{https://github.com/cjhang}{\underline{github.com/cjhang}} \\
    Max Planck Institute for Extraterrestrial Physics \\
    Giessenbachstrasse 1\\ 85748 Garching, Munich, Germany
    %European Southern Observatory \\
    %Karl-Schwarzschild-Strasse 2\\ 85748, Garching bei München, Germany
\end{minipage}

% Without picture
%\begin{center}
%    \textbf{\Huge \scshape Charles Rambo} \\ \vspace{1pt} %\scshape sets small capital letters, remove if desired
%    \small +1 123-456-7890 $|$ 
%    \href{mailto:you@provider.com}{\underline{you@provider.com}} $|$\\
%    % Be sure to use a professional *personal* email address
%    \href{https://linkedin.com/in/your-name-here}{\underline{linkedin.com/in/charles-rambo}} $|$
%    % you should adjust you linked in profile name to be professional and recognizable
%    \href{https://github.com/fizixmastr}{\underline{github.com/fizixmastr}}
%\end{center}



\begin{comment}
This CV was written for specifically for positions I was applying for in
academia, and then modified to be a template.

A standard CV is about two pages long where as a resume in the US is one page.
sections can be added and removed here with this in mind. In my experience, 
education, and applicable work experience and skills are the most import things
to include on a resume. For a CV the Europass CV suggests the categories: Work
Experience, Education and Training, Language Skills, Digital Skills,
Communication and Interpersonal Skills, Conferences and Seminars, Creative Works
Driver's License, Hobbies and Interests, Honors and Awards, Management and
Leadership Skills, Networks and Memberships, Organizational Skills, Projects,
Publications, Recommendations, Social and Political Activities, Volunteering.

Your goal is to convey a who, what , when, where, why for every item you share. 
The who is obviously you, but I believe the rest should be done in that order.
For example below. An employer cares most about the degree held and typically 
less about the institution or where it is located (This is still good 
information though). Whatever order you choose be consistent throughout.
\end{comment}

%-----EDUCATION----------------------------------------------------------------
\section{Education}
  \CVSubHeadingListStart
%    \CVSubheading % Example
%      {Degree Achieved}{Years of Study}
%      {Institution of Study}{Where it is located}
    \CVSubheading
    {{PhD of Astronomy $|$ {\small{ESO \& Ludwig Maximilian University of Munich}}}}{Sep. 2020 -- Aug. 2023}
     {\emph{Distant, dusty star-forming galaxies} $|$ Supervisor: Prof. Rob Ivison}{Garching, Germany}
    \CVSubheading
      {{Master of Astrophysics $|$ {\small{Nanjing University}}}}{Sep. 2017 -- Jun. 2020}
      {\emph{Studying galactic feedbacks with multi-phase gas} $|$ Supervisor: Prof. Yong Shi \& Prof. Zhi-yu Zhang}{Nanjing, China}
    \CVSubheading
      {{Bachelor of Physics $|$ {\small{Lanzhou University}}}}{Sep. 2013 -- May. 2017}
      {\emph{}}{Lanzhou, China}
  \CVSubHeadingListEnd

%-----WORK EXPERIENCE----------------------------------------------------------
\section{Employment}
  \CVSubHeadingListStart
    \CVSubheading
    {{Postdotoral Fellow $|$ {\small{Max Planck Institute for Extraterrestrial Physics}}}}{Sep. 2023 -- now}
     {\emph{}}{Garching, Germany}
  \CVSubHeadingListEnd


\section{Telescope Proposals and Observing Experience}
{\bf PI Proposals}:\\
\vspace{-0.8em}
\begin{paperlist}
  \item {\bf ALMA} Cycle-12, 2025.1.00710.S: \emph{Do Normal Galaxies Host Ordered Magnetic Fields at Cosmic Noon?}, {\bf 27\,hrs (A priority)}. 
  \item {\bf ALMA} Cycle-12, 2025.1.00635.S: \emph{A Magnetic Rosetta Stone: measuring magnetic fields in a distant starburst at z=2.6}, {\bf 37\,hrs (B priority)}. 
  \item {\bf VLBA} VLBA/25B-191, \emph{Probing the jet-BLR connection in an early accreting efficient SMBH}, {\bf 8+12\,hrs (B priority)}.
  \item {\bf VLT/ERIS} P114, 114.271N: \emph{What drives the quick formation of galactic magnetic fields at cosmic noon?}, {\bf 10.6\,hrs (B priority)}
  \item {\bf NOEMA} Winter 2023, W23CR: \emph{Into the heart of darkness}, {\bf 32\,hr (B priority)}.
  \item {\bf ALMA} Cycle-10, 2023.1.00318.S: \emph{How important are magnetic fields in early disk galaxies?}, {\bf 30.4\,hrs (B priority)}. 
  \item {\bf VLT/MUSE} P111, 111.253V: \emph{Probing unobscured star formation in a new sample of ALMA-selected proto-cluster cores at $z>3$}, {\bf 16\,hrs (B priority)}.
  \item {\bf IRAM30m} Summer 2022 Delta, 095-22: \emph{The tip of the iceberg?}, {\bf 34.4\,hrs (A priority)}.
  \item {\bf ALMA} Cycle-9, 2022.1.00495.S: \emph{Isotopic constraints on the IMF in the most extreme star-forming environments in the Universe}, {\bf 23.7\,hrs (A priority)}. 
  \item {\bf JCMT} 22A, M22AP002: \emph{Revealing the structure around a putative new SMG proto-cluster core}, {\bf 8.5\,hrs (B priority)}.
  \item {\bf ALMA} Cycle-8, 2021.1.00458.S: \emph{Dust polarisation in distant, star-foming galaxies: a transformative survey of the viable targets}, {\bf 25.1\,hrs (B priority)}. 
  \item {\bf NEOMA} Summer 2020, S20BB: \emph{Searching for molecular gas in HI-rich red spirals}, {\bf 2.8\,hrs (B priority)}. 
\end{paperlist}

{\bf Selected Co-I Proposals (leading contribution)}:\\
\vspace{-0.8em}
\begin{paperlist}
    \item {\bf ALMA} Cycle-12, 2025.1.00878.S: \emph{Revealing the B-fields at ~800 Myr after the Big Bang}, 14\,hrs, (PI. Enrique Lopez-Rodriguez, A priority).
    \item {\bf ALMA} Cycle-12, 2025.1.00878.S: \emph{Pilot survey tracing the magnetic fields across z=1.9-4.5 using dusty star-forming galaxies}, 49\,hrs, (PI. Enrique Lopez-Rodriguez).
    \item {\bf ALMA} Cycle-12, 2025.1.00494.S: \emph{The first resolved study of dust polarization in a sample of extreme starbursts}, 30.6\,hrs, (PI. Kevin Harrington, A priority).
    \item {\bf VLA} VLA/25A-165, \emph{THE corona hunt in high-z lensed quasars} 8\,hrs, (PI Santiago Del Palacio)
    \item {\bf ALMA} Cycle-8, 2021.1.00018.S: \emph{Exploiting a snapshot survey of the 3,083 reddest Herschel sources to reveal distant protoclusters}, 30.6\,hrs, (PI. R. Ivison).
    \item {\bf VLA} 2020B, 20B-180: \emph{Off-nuclear star formation in a gas-rich low-mass S0 galaxy}, 27\,hrs, (PI: Zhengyi Chen)
    \item {\bf NOEMA} Summer 2020, S20CK: \emph{Exploiting a snapshot survey of the 3,083 reddest Herschel sources to reveal distant protoclusters}, 26\,hrs, (PI: Vinodiran Arumgam)
    \item {\bf IRAM30m} Summer 2020, 062-20: \emph{Do Changing-look AGNs reside in the gas-rich environment?}, 23.5\,hrs, (PI: Xiaoling Yu)
\end{paperlist}

{\bf Observing Experience}:\\
\vspace{-0.8em}
\begin{paperlist}
    \item IRAM30m, NIKA2, 6 nights, 18-25 Oct. 2022
    \item IRAM30m, NIKA2, 7 nights, 13-21 Oct. 2024
    \item Paranal, VLT/ERUS, 6 half-nights, 24-31 Aug. 2024
    \item Paranal, VLT/ERUS, 6 half-nights, 26-31 Aug. 2025
\end{paperlist}


%-----PROJECTS AND RESEARCH----------------------------------------------------
\begin{comment}
Ideally the title of the work should speak for what it is. However if you feel
like you should explain more about why the project is applicable to this job,
use item list as is shown in the work experience section.
\end{comment}

%-----RESEARCH EXPERIENCE------------------------------------------------------
\section{Research Experience}
  \CVSubHeadingListStart
    \CVSubheading 
      {Short term visit, Durham University/Physics department}{14-24 Sep. 2022}
      {Host: Prof. Mark Swinbank Topic: dynamical modeling}{Durham, UK}
    \CVSubheading
      {11th NOEMA Interferometry School}{21-25 Nov.~2021}
      {Winter School}{Grenoble, France}
    \CVSubheading 
      {Astrophysics, Formation and Evolution of Galaxy cluster Across Cosmic Time}{23 Nov.-1 Dec.~2021}
      {Winter School}{Tenerife, Spain}
    \CVSubheading 
      {10th IRAM 30-meter School on Millimeter Astronomy}{15-23 Nov.~2021}
      {Winter School}{Online}
    \CVSubheading
      {Astrostatistics and R training}{09-13 Jul.~2018}
      {The 2nd East Asian Astrostatistics International Conference (EAAIC2018)}{Nanjing, China}
  \CVSubHeadingListEnd


\section{Conference and Seminar Talks}
  \CVSubHeadingListStart
%    \CVSubheading
%      {Title of Work}{When it was done}
%      {Institution you worked with}{unused}
    \CVSubheading
      {Dusty, star-forming galaxies}{11~Jan. 2023}
      {ASIAA Colloquium}{Remote}
    \CVSubheading
      {Opportunities with dusty, star-forming galaxies}{9~Jan. 2023}
      {MPE Seminar talk}{Garching, Germany}
    \CVSubheading
      {ALMACAL IX: multi-band ALMA survey for dusty star-forming galaxies}{15--23~Dec. 2022}
      {A half-century of millimeter and submillimeter astronomy}{Miyakojima, Japan}
    \CVSubheading
      {ALMACAL: Number counts and the resolved fractions of the CIB}{3--4~Mar. 2022}
      {Meeting of ALMA Young Astronomers}{Online}
    \CVSubheading
      {Submillimetre number counts}{5--6~Nov. 2021}
      {12th IMPRS Symposium}{Garching, Germany}
    \CVSubheading
      {The spatial extension of AGN narrow line regions}{1--4 Apr. 2019}
      {MaNGA Collaboration Meeting}{Oxford, UK}
  \CVSubHeadingListEnd

%-----CONFERENCES AND PRESENTATIONS--------------------------------------------
\begin{comment}
Again the title should have already been enough, but if it is necessary to add
descriptions maintain the consistency from prior sections
\end{comment}

\section{Posters}
  \CVSubHeadingListStart
%    \CVSubheading % Example
%      {Work Presented}{When}
%      {Occasion}{}
    \CVSubheading
      {ALMACAL: A `free' submillimetre galaxy survey}{3--5~May. 2022}
      {PoSTER 2022 - Galaxy Evolution (Best poster award)}{Online}
    \CVSubheading
      {ALMACAL: Hunting for regions over-dense in SMGs}{23~Nov.--1~Dec. 2021}
      {XXXII Canary Islands Winter School of Astrophysics}{Tenerife, Spain}
  \CVSubHeadingListEnd



%-----TEACHING EXPERIENCE------------------------------------------------------
\begin{comment}
Section is here as it applied to my application for positions in academia. 
Remember to tailor the resume for to the position.
\end{comment}

%-----COMMUNITY INVOLVEMENT----------------------------------------------------
\section{Community Involvement}
  \CVSubHeadingListStart
%    \CVSubheading %Example
%      {What you did}{When you worked there}
%      {Who you worked for}{Where they are located}
    \CVSubheading
      {Scientific Assistant at ESO OPC P109 \& P110}{Nov. 2021 \& May. 2022}
      {Helping to organise pannel discussion for time allocation of ESO telescopes}{}
    \CVSubheading
      {Organisor of student session of Munich Joint Astronomy Colloquium}{Nov. 2021}
      {Helping to coodinate the discussion between students and the speaker}{}
    \CVSubheading
      {Member of LOC for the 11th IMPRS Symposium}{May 2021}
      {Helping to organise meeting schedule}{}

    \CVSubheading
      {Scientific referee for Publications of the Astronomical Society of Japan}{2020}
      {}{}
  \CVSubHeadingListEnd

%-----HONORS AND AWARDS--------------------------------------------------------
\section{Awards and Scholarship}
  \CVSubHeadingListStart
%    \CVSubheading %Example
%      {What}{When}
%      {Short Description}{}
    \CVSubheading
      {International Max-Planck Research Scholarship (IMPRS) on Astrophysics}{2020-2023}
      {Grant for PhD study}{}
    \CVSubheading
      {Excellence Scholarship}{Spring 2019}
      {Nanjing University}{}
    \CVSubheading
      {Outstanding Graduates Award}{Spring 2017}
      {Lanzhou University}{}
    \CVSubheading
      {National Encouragement Scholarship}{2015,2016}
      {Merit based scholarship for colleage students in China}{}
    \CVSubheading
      {The First Prize Scholarships}{Spring 2014}
      {Lanzhou University}{}
  \CVSubHeadingListEnd


%-----SKILLS-------------------------------------------------------------------
\begin{comment}
This section is compressed from the various skills sections that Euro CV
recommends.
\end{comment}

\section{Skills}
 \begin{itemize}[leftmargin=0.5cm, label={}]
    \small{\item{
     \textbf{Languages}{: Chinese (Native), English (C1)} \\
     \textbf{Programming}{: Python (NumPy, SciPy, Matplotlib, Pandas), MATLAB, Mathematica} \\
     \textbf{Document Creation}{: HTML, LaTex, Markdown} \\
     \textbf{Software}{: pPXF, CASA, CARTA, DS9, BBarolo, Galpak}
    }}
 \end{itemize}

\section{References}

\begin{minipage}[c]{0.05\textwidth}
\-\
\end{minipage}
\begin{minipage}[t]{0.42\textwidth}
$\bullet$ \ Prof.\,Dr.\,Rob Ivison \\
\hphantom{$\bullet$} \ European Southern Observatory (ESO) \\
\hphantom{$\bullet$} \ \href{mailto:Rob.Ivison@eso.org}{Rob.Ivison@eso.org} 
\vspace{1em}
\\
%\end{minipage}
%\begin{minipage}[c]{0.5\textwidth}
$\bullet$ \ Dr.\,Martin Zwaan \\
\hphantom{$\bullet$} \ ALAM Regional Center / ESO \\
\hphantom{$\bullet$} \ \href{mailto:mzwaan@eso.org}{mzwaan@eso.org} 
\end{minipage}
\begin{minipage}[t]{0.52\textwidth}
$\bullet$ \ Dr.\,C\'eline P\'eroux \\
\hphantom{$\bullet$} \ European Southern Observatory\\
\hphantom{$\bullet$} \ Aix Marseille Universit\'e, CNRS, LAM\\
\hphantom{$\bullet$} \ \href{mailto:celine.peroux@gmail.com}{celine.peroux@gmail.com}
\end{minipage}




\newpage

\begin{table*}
  \centering
  \begin{tabular*}{0.9\textwidth}[t]{l@{\extracolsep{\fill}}r}
    {\Huge Jianhang CHEN} & {\huge\sc List of Publications} \\
    Max Planck Institute for Extraterrestrial Physics & jhchen@mpe.mpg.de \\
    Giessenbachstrasse 1 & \\
    85748, Garching bei München, Germany & \href{https://ui.adsabs.harvard.edu/public-libraries/F2VsSSiDSeuhqtOkcg-qVg}{\underline{ADS Library}}
  \end{tabular*}
\end{table*}

%\par\noindent\rule{\textwidth}{0.5pt}
%\vspace{-3em}

\section{First \& second author papers:}
%\vspace{-0.8em}
\begin{paperlist}
  \item ``SDSS-IV/MaNGA: The Size–Luminosity Relation of Extended Narrow Line Region in Low-luminosity AGNs Xu'' {\bf Chen}, et al., RAA, 25, 115016 (2025)
  \item ``A kiloparsec-scale ordered magnetic field in a galaxy at z = 5.6'' {\bf Chen}, Lopez-Rodriguez, Ivison et al. A\&A, 692A, 34 (2024).
  \item ``ALMACAL. XI. Over-densities as signposts for proto-clusters? A cautionary tale'' {\bf Chen}, Ivison, Zwaan et al. A\&A 675L, 10 (2023). 
    \item ``ALMACAL IX: multi-band ALMA survey for dusty star-forming galaxies and the resolved fractions of the cosmic infrared background" {\bf Chen}, Ivison, Zwaan et al. MNRAS, 518, 1378 (2023)
    \item ``The spatial extension of extended narrow line regions in MaNGA AGN" {\bf Chen}, Shi, Dempsey et al. MNRAS, 489, 855 (2019)
\end{paperlist}

\section{Publicaons in preparaon and selected non-refereed publicaons:}
\begin{paperlist}
    \item ``NOEMA3D. Spatially resolved Dust, CO, and [C\,I] in massive star-forming main sequence galaxies at cosmic noon'' {\bf Chen}, NOEMA3D Team (2026). 
    \item ``MUSE-ALMA Haloes XIV. The ALMA Large Programme Data Release'' P\'eroux, {\bf Chen}, Bollo et al.
    \item ``Licking the plate: Dusty star-forming galaxies buried in the ALMA calibration data'' {\bf Chen},  Ivison, Zwaan et al. EPJ Web of Conferences, Volume 293, id.00011 
\end{paperlist}

\section{Co-author papers:}
%\par\noindent\rule{\textwidth}{0.5pt}\\
%\vspace{-0.8em}
% \item ``%T'' %4m, %\q, %\V, %\p (%Y) 

\begin{paperlist}

  \item ``Probing Infrared eXcess to Investigate Early-Universe Dust (PIXIEDust)'' Bakx et al., arXiv, arXiv:2512.07964 (2025)
\item ``Resolving stellar populations, star formation, and ISM conditions with JWST in a large spiral galaxy at z $\sim$ 2'' Parlanti et al., arXiv, arXiv:2510.09820 (2025)
\item ``The ALMA-CRISTAL survey: Resolved kinematic studies of main sequence star-forming galaxies at 4 < z < 6'' Lee et al., A\&A, 701, A260 (2025)
\item ``PHIBSS: Searching for Molecular Gas Outflows in Star-forming Galaxies at z = 0.5–2.6'' Barfety et al., ApJ, 988, 55 (2025)
\item ``A Comparative Study of the Ground State Transitions of CO and C I as Molecular Gas Tracers at High Redshift'' Frias Castillo et al., ApJ, 987, 158 (2025)
\item ``The Ultradiffuse Galaxy AGC 242019 with a Negative Metallicity Gradient'' Ni et al., ApJ, 986, 112 (2025)
\item ``Galaxy morphologies at cosmic noon with JWST: A foundation for exploring gas transport with bars and spiral arms'' Espejo Salcedo et al, A\&A, 700A, 42 (2025)
\item ``NOEMA$^{\rm 3D}$: A first kpc resolution study of a $z\sim1.5$ main sequence barred galaxy channeling gas into a growing bulge'' Pastras et al., arXiv, arXiv:2505.07925 (2025)
\item ``ALMACAL – XIV. X-Shooter spectroscopy, infrared properties, and radio SEDs of calibrators'' Weng et al., MNRAS, 539, 1977 (2025)
\item ``Deep kiloparsec view of the molecular gas in a massive star-forming galaxy at cosmic noon'' Arriagada-Neira et al., A\&A, 696, A83 (2025)
\item ``ALMACAL: XIII. Evolution of the CO luminosity function and the molecular gas mass density out to z ∼ 6'' Bollo et al., A\&A, 695, A163 (2025)
\item ``Unveiling Cosmic Cold Gas: Insights from ALMACAL survey'' Bollo et al., rcmi.conf, 20 (2024)
\item ``ALMACAL: XII. Data characterisation and products'' Bollo et al., A\&A, 690, A258 (2024)
\item ``Detailed study of a rare hyperluminous rotating disk in an Einstein ring 10 billion years ago'' Liu et al., NatAs, 8, 1181 (2024)

    \item ``Polarized thermal emission from dust in a galaxy at redshift 2.6'' Geach et al. (4th co-author) Nature, 621, 483 (2023)
    \item ``An escaping outflow in a galaxy with an intermediate-mass black hole'' Zheng et al. MNRAS, 523, 3274 (2023)
    \item ``VLA Legacy Survey of Molecular Gas in Massive Star-forming Galaxies at High Redshift'' Frias Castillo et al., ApJ, 945, 128 (2023)
    \item ``The H I gas disc thickness of the ultra-diffuse galaxy AGC 242019'' Li et al. (4th co-author) MNRAS, 516, 4220 (2022)
    \item ``ALMACAL: Surveying the Universe with ALMA Calibrator Observations'' Zwaan et al. (4th co-author) The Messenger, 186, 10 (2022)
    \item ``The major mechanism to drive turbulence in star-forming galaxies'' Yu et al. MNRAS, 505, 5075 (2021)
    \item ``Probing possible effects of circumgalactic media on the metal content of galaxies through the mass-metallicity relationship'' Zhai et al. (3th co-author) MNRAS, 504, 1959 (2021)
    \item ``A Cuspy Dark Matter Halo” Shi, Zhang, Wang et al. (4th co-author) ApJ, 909, 20 (2021)
    \item ``The impact of merging on the origin of kinematically misaligned and counter-rotating galaxies in MaNGA'' Li et al. (8th co-author) MNRAS, 501, 14 (2021)
    \item ``Host galaxy properties of changing-look AGNs revealed in the MaNGA survey" Yu et al. (4th co-author) MNRAS, 498, 3985 (2020)
    \item ``What drives the velocity dispersion of ionized gas in star-forming galaxies?" Yu et al. MNRAS, 486, 4463 (2019)
    \item ``An early-type galaxy with an inner star-forming disc" Li et al. MNRAS, 480, 1705 (2018)
\end{paperlist}
%------------------------------------------------------------------------------
\end{document}
